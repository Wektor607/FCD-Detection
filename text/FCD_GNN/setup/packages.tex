\usepackage[utf8]{inputenc}
\usepackage[T1]{fontenc}
\usepackage[english]{babel} % sets up english hyphenation
\usepackage{csquotes} % for language-dependent quotes in biblatex
\usepackage{eurosym} %includes the euro symbol 
\usepackage[dvipsnames, hyperref]{xcolor} % enables more advanced color support for hyperref

\usepackage{acro}

\usepackage{graphicx} % enables loading of graphics
\makeatletter
\def\maxwidth{\ifdim\Gin@nat@width>\linewidth \linewidth \else \Gin@nat@width\fi}
\makeatother

\graphicspath{ {images/}{../images/}{guidelines/images/}}

\usepackage[
  backend=biber,
  style=numeric-comp,  % числовые ссылки [1], с компрессией диапазонов
  sorting=none,        % порядок в списке – как в тексте
  maxbibnames=100,
  giveninits,
  backref=true,        % ссылки из библиографии назад в места цитирования
  doi=false, isbn=false, url=false, eprint=false
]{biblatex}
\addbibresource{literature.bib} % Add a bibliography file. (\bibliography{bib file} is less flexible and should not be used anymore although its still supported by biblatex)

\usepackage{subfiles} % Use this package if you want to seperately compile child documents

% Create nice tables in your document
\usepackage{pdflscape} % Provides the landscape environment for landscape tables
\usepackage{tabu}     % provides advanced tables
\usepackage{float}
\usepackage{array,multirow}
\usepackage{booktabs} % enables reference bookstyle tables
\usepackage{longtable} % Make multipage tables without creating a new table for every page
\usepackage[table]{xcolor} % поддержка раскраски ячеек
\usepackage{threeparttable}
\usepackage[most]{tcolorbox}
\tcbset{
  myhighlight/.style={
    colback=green!20,          % светло-зелёный фон
    colframe=green!50!black,   % более тёмная рамка
    boxrule=0pt,               % без рамки (0pt)
    arc=2mm,                   % скруглённые углы
    left=2mm, right=2mm, top=1mm, bottom=1mm
  }
}
\usepackage{enumitem}
\usepackage[format=plain, labelfont=bf]{caption} % один раз!
\usepackage{subcaption} % для подфигур
\captionsetup{compatibility=false}

% Настройки отступов для таблиц
\captionsetup[table]{skip=2pt} % расстояние между caption и таблицей
\setlength{\textfloatsep}{10pt plus 1pt minus 2pt} % отступ сверху/снизу для table/figure
\setlength{\floatsep}{8pt plus 1pt minus 2pt}      % отступ между плавающими объектами

%\usepackage{parskip} %alternatively parskip replaces paragraph indentation by increased in-betweeen-paragraph linespacing 
\usepackage{setspace} % helps setup line spacing
%\onehalfspacing % increases linespacing to one and half
\usepackage{placeins} % provides FloatBarrier
\usepackage[ruled,vlined]{algorithm2e} %algorithm package
\linespread{1.1} % Definition of the linespread

\usepackage[tbtags]{mathtools}
\DeclareMathOperator*{\somefunc}{somefunc}
     
\usepackage[unicode=true]{hyperref}
\hypersetup{
  colorlinks=true, % цветные ссылки
  linkcolor=red!35!black,
  citecolor=green!35!black,
  urlcolor=magenta!35!black,
  pdfauthor={Mikhelson German},
  pdftitle={Detection of Focal Cortical Dysplasia Type II Using Text Descriptions}
}

\usepackage[capitalize,noabbrev]{cleveref}
\usepackage{amsmath,amssymb} % для align, \mathbb и т.п.
\usepackage{bbm}
% Specify headers and footers in different file for better overview
\input{setup/headfoot.tex}
\usepackage{makecell} % в преамбуле
\usepackage[a4paper,margin=2cm]{geometry}