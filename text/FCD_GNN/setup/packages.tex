\usepackage[utf8]{inputenc}
\usepackage[T1]{fontenc}
\usepackage[english]{babel} % sets up english hyphenation
\usepackage{csquotes} % for language-dependent quotes in biblatex
\usepackage{eurosym} %includes the euro symbol 
\usepackage[dvipsnames, hyperref]{xcolor} % enables more advanced color support for hyperref

\usepackage{acro}

\usepackage{graphicx} % enables loading of graphics
\graphicspath{ {images/}{../images/}{guidelines/images/}}

\usepackage[
  backend=biber,
  style=numeric-comp,  % числовые ссылки [1], с компрессией диапазонов
  sorting=none,        % порядок в списке – как в тексте
  maxbibnames=100,
  giveninits,
  backref=true,        % ссылки из библиографии назад в места цитирования
  doi=false, isbn=false, url=false, eprint=false
]{biblatex}
\addbibresource{literature.bib} % Add a bibliography file. (\bibliography{bib file} is less flexible and should not be used anymore although its still supported by biblatex)

\usepackage{subfiles} % Use this package if you want to seperately compile child documents

% Create nice tables in your document
\usepackage{pdflscape} % Provides the landscape environment for landscape tables
\usepackage{tabu}     % provides advanced tables
\usepackage{array,multirow}
\usepackage{booktabs} % enables reference bookstyle tables
\usepackage{longtable} % Make multipage tables without creating a new table for every page

\usepackage[format=plain, labelfont=bf]{caption}
\usepackage{subcaption} % enables use of multiple figures in a figure
\captionsetup{compatibility=false}
\usepackage{enumitem} % allows customization of enumeration and itemize environment

%\usepackage{parskip} %alternatively parskip replaces paragraph indentation by increased in-betweeen-paragraph linespacing 
\usepackage{setspace} % helps setup line spacing
%\onehalfspacing % increases linespacing to one and half
\usepackage{placeins} % provides FloatBarrier
\usepackage[ruled,vlined]{algorithm2e} %algorithm package
\linespread{1.1} % Definition of the linespread

\usepackage[tbtags]{mathtools}
\DeclareMathOperator*{\somefunc}{somefunc}
     
\usepackage[unicode=true]{hyperref}
\hypersetup{
  colorlinks=true, % цветные ссылки
  linkcolor=red!35!black,
  citecolor=green!35!black,
  urlcolor=magenta!35!black,
  pdfauthor={Mikhelson German},
  pdftitle={Detection of Focal Cortical Dysplasia Type II Using Text Descriptions}
}

\usepackage[capitalize,noabbrev]{cleveref}

% Specify headers and footers in different file for better overview
\usepackage{fancyhdr} % define nice headers and footers easily

% Some examples for customizing your headers and footers, you may also want to read this:
% http://tug.ctan.org/macros/latex/contrib/fancyhdr/fancyhdr.pdf

% Define or redefine pagestyles
\fancypagestyle{thesis}{ % The same as the fancy pagestyle
	\fancyhead[LE,RO]{\textsl{\rightmark}}
	\fancyhead[LO,RE]{\textsl{\leftmark}}
	\fancyfoot[C]{\thepage}
	
	\renewcommand{\headrulewidth}{0.4pt}
	\renewcommand{\footrulewidth}{0pt}}

\fancypagestyle{bib}{ %
	\fancyhf{} % clean all fields (header and footer)	
	\fancyhead[LE,RO]{References}
	\fancyfoot[C]{\thepage}
	
	\renewcommand{\headrulewidth}{0.4pt}
	\renewcommand{\footrulewidth}{0pt}}

% Redefine plain pagestyle to put pagenumber to the center for the first page of every chapter
\fancypagestyle{plain}{
	\renewcommand{\headrulewidth}{0pt}
	\fancyhf{}
	\fancyfoot[C]{\thepage}} 

\pagestyle{fancy} % use the default layout

