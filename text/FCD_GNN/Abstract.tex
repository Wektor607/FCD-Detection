\documentclass[FCD_GNN.tex]{subfiles}

\begin{document}
\chapter*{Abstract}
\addcontentsline{toc}{chapter}{Abstract}

Numerous methods have been developed for the detection of tumors in internal organs such as the lungs, brain, kidneys, 
and breast \parencite{Durgam2025LungCancer, Abdusalomov2023BrainTumor, Sharma2025KidneyCT, Mehmood2025BreastCancer}. 
However, detecting epileptogenic lesions remains significantly more challenging. Unlike tumors, these lesions do not typically 
increase in size over time, and there is a severe shortage of publicly available annotated datasets. Researchers often need 
to contact hospitals and clinical centers directly to obtain even a minimal number of scans. Although recent studies have 
demonstrated that deep learning models can detect epileptogenic lesions \parencite{Ripart2025MELD}, their performance in terms of Intersection over 
Union (\acs{iou}) rarely exceeds 35\%, highlighting the ongoing difficulty of this task.

At the same time, recent work on tumor detection using text-guided approaches has shown promising results \parencite{Li2024LGMSeg, Li2025Mulmodseg, Zhang2025OrganAware}. 
Notably, \parencite{Zhong2023Ariadne} demonstrated that training on only 10\% of the dataset can yield performance comparable to using the full dataset. 
Inspired by these advances, this study proposes a new method that combines visual and textual features to improve Focal Cortical Dysplasia detection under limited data conditions. 
The proposed approach further demonstrates that incorporating textual descriptions significantly enhances segmentation accuracy. 
We present a comparison of multiple types of textual annotations and analyze their influence on model performance.

% \thispagestyle{empty}
\end{document}
