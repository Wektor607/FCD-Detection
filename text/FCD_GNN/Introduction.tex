\documentclass[FCD_GNN.tex]{subfiles}

\begin{document}
\chapter{Introduction}
Automatic detection of tumors using medical imaging is widely studied in many internal organs. 
Recent advances in deep learning and transformer-based approaches have led to impressive results in the diagnosis of lung cancer~\cite{Durgam2025LungCancer}, brain tumor~\cite{Abdusalomov2023BrainTumor}, kidney \ac{ct}~\cite{Sharma2025KidneyCT}, and breast cancer~\cite{Mehmood2025BreastCancer}. 
These studies demonstrate the potential of modern computer vision techniques to achieve clinically significant results in various diagnostic tasks.  

On the contrary, the detection of epileptogenic lesions, in particular \ac{fcd}, remains a much more difficult task. 
Unlike tumors, such lesions usually do not change in size over time, and their inconspicuous appearance makes it difficult even for experienced neuroradiologists to identify them. 
In addition, the lack of large publicly available annotated datasets is hindering progress in this area. 
To address this problem, within the framework of the \ac{meld} project a large-scale collaborative dataset has recently been created and graph neural network–based approaches have been developed for epileptogenic lesion detection~\cite{Ripart2025MELD}. 
Although this is the most advanced solution to date, performance remains limited, underscoring the continued complexity of this task.

At the same time, \textit{text-guided} and \textit{multimodal approaches} have gained momentum in medical image analysis. 
Integrating textual prompts or language-guided embeddings with visual features has been shown to substantially enhance segmentation accuracy in chest X-ray infection detection~\cite{Zhong2023Ariadne}, 
language-guided multi-level alignments~\cite{Li2024LGMSeg}, organ-aware segmentation~\cite{Zhang2025OrganAware}, multimodal tumor analysis~\cite{Li2025Mulmodseg}, 
and pneumothorax segmentation~\cite{Huemann2024ConTEXTualNet}. 
These advances demonstrate that textual annotations, ranging from atlas-based region names to descriptive clinical reports, provide valuable complementary information. 
By projecting textual and visual features into a shared latent space, such information can be effectively aligned and leveraged to enhance model performance.

We systematically evaluate the impact of different types of textual information, as well as alternative strategies 
for combining visual features, including the integration of an additional \ac{gnn} blocks on top of \ac{meld}-based representations. 
We analyze how these design choices influence segmentation performance, demonstrating that the joint integration 
of language and visual features can enhance both the accuracy and the sensitivity of epileptogenic lesion detection.

\end{document}
