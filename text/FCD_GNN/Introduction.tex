\documentclass[FCD_GNN.tex]{subfiles}

\begin{document}
\chapter{Introduction}

This chapter first highlights the importance of epilepsy detection and discusses the limitations of existing methods. 
It then outlines the main objectives of the thesis and its research contributions. 
Finally, the chapter concludes with an overview of the thesis structure.

\section{Motivation}
Automatic detection of tumors using medical imaging is widely studied in many internal organs. 
Recent advances in deep learning and transformer-based approaches have led to impressive results in the diagnosis of lung cancer~\cite{Durgam2025LungCancer}, brain tumor~\cite{Abdusalomov2023BrainTumor}, kidney \ac{ct}~\cite{Sharma2025KidneyCT}, and breast cancer~\cite{Mehmood2025BreastCancer}. 
These studies demonstrate the potential of modern computer vision techniques to achieve clinically significant results in various diagnostic tasks.  

On the contrary, the detection of epileptogenic lesions, in particular \ac{fcd}, remains a much more difficult task. 
Unlike tumors, such lesions usually do not change in size over time, and their inconspicuous appearance makes it difficult even for experienced neuroradiologists to identify them. 
In addition, the lack of large publicly available annotated datasets is hindering progress in this area. 
To address this problem, within the framework of the \ac{meld} project a large-scale collaborative dataset has recently been created and graph neural network–based approaches have been developed for epileptogenic lesion detection~\cite{Ripart2025MELD}. 
Although this is the most advanced solution to date, performance remains limited, underscoring the continued complexity of this task.

At the same time, \textit{text-guided} and \textit{multimodal approaches} have gained momentum in medical image analysis. 
Integrating textual prompts or language-guided embeddings with visual features has been shown to substantially enhance segmentation accuracy in chest X-ray infection detection~\cite{Zhong2023Ariadne}, 
language-guided multi-level alignments~\cite{Li2024LGMSeg}, organ-aware segmentation~\cite{Zhang2025OrganAware}, multimodal tumor analysis~\cite{Li2025Mulmodseg}, 
and pneumothorax segmentation~\cite{Huemann2024ConTEXTualNet}. 
These advances demonstrate that textual annotations, ranging from atlas-based region names to descriptive clinical reports, provide valuable complementary information. 
By projecting textual and visual features into a shared latent space, such information can be effectively aligned and leveraged to enhance model performance.

\section{Aims of the Thesis}
The main objective of this thesis is to investigate how the integration of textual and visual information 
can improve the accuracy of epileptogenic lesion detection. 
To this end, several research questions were formulated:

\begin{enumerate}
    \item Does incorporating textual descriptions from anatomical atlases influence the accuracy of lesion segmentation?
    \item Can competitive performance be achieved using only partial atlas information (e.g., hemisphere or lobe labels)?
    \item How does the integration of additional \ac{gnn} blocks on top of \ac{meld}-based representations affect segmentation quality?
\end{enumerate}

We systematically evaluate the impact of different types of textual information, as well as alternative strategies 
for combining visual features, including the integration of additional \ac{gnn} layers. 
We further analyze how these design choices influence segmentation performance, 
demonstrating that the joint integration of language and visual features can enhance both the accuracy 
and the sensitivity of epileptogenic lesion detection.

\section{Contributions}

The main contributions of this thesis can be summarized as follows:

\begin{enumerate}
    \item Proposes a novel multimodal architecture for the detection of epileptogenic lesions, c
    ombining visual representations with language-guided features.
    \item Conducted a comprehensive set of experiments to explore design choices, including the number of 
    connections in the \ac{gnn} blocks and the integration of different textual branches in the GuideDecoder, 
    in order to optimize segmentation performance.
    \item Systematically investigated the effect of different types of textual descriptions (e.g., 
    full atlas annotations, hemispheric information, lobe-level labels) on model accuracy and sensitivity.
    \item Provided thorough documentation and usage guidelines for the developed code.
\end{enumerate}

\section{Thesis structure}
Chapter~\ref{chapter:Related_works} reviews prior research on \acs{fcd} detection and text-guided medical image segmentation, highlighting key architectural designs, fusion strategies, and limitations that motivate our approach. 
Chapter~\ref{chapter:Method} introduces the proposed multimodal segmentation framework for epileptogenic lesion detection, describing its surface-based visual feature extraction, GNN block, RadBERT textual embeddings, and their integration within the GuideDecoder. 
Chapter~\ref{chapter:Dataset} details the datasets used in this study, namely the Bonn dataset and the large-scale MELD cohort, along with the preprocessing steps, data augmentation strategies, and types of atlas-based textual descriptions. 
Chapter~\ref{chapter:Experiments} TODO

\end{document}
