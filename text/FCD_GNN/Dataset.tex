\documentclass[FCD_GNN.tex]{subfiles}

\begin{document}
\chapter{Dataset}
\label{chapter:Dataset}
We gratefully acknowledge the \textbf{\ac{meld} Project} for providing access to the dataset used in this work. 
Without this resource, it would not have been possible to conduct a systematic evaluation of our model.

\section{Bonn Dataset}

The Bonn scientific group has published a presurgical MRI dataset on \textbf{OpenNeuro}, entitled 
“\textit{An open presurgery MRI dataset of people with epilepsy and focal cortical dysplasia type II}”~\cite{Rueber2023_BonnFCDDataset}.  
This dataset comprises \textbf{170 participants}, including \textbf{85 individuals with FCD type II} and \textbf{85 healthy controls}. 
For each participant, the following data are available:

\begin{itemize}
    \item High-resolution \textbf{3D T1-weighted} MRI scans (isotropic voxels, 1\,mm$^{3}$ or 0.8\,mm$^{3}$ resolution, depending on the subject);
    \item Corresponding \textbf{isotropic 3D FLAIR} imaging;
    \item \textbf{Manually delineated regions of interest (ROIs)} identifying FCD lesions, provided for patients only;
    \item A set of \textbf{clinical and demographic variables} (including age, sex, lesion laterality and location, histopathological subtype IIa/IIb, MRI-negative status, and postoperative outcome according to Engel classification).
\end{itemize}

\subsection*{Dataset Structure}
The dataset follows the BIDS (Brain Imaging Data Structure) convention, with each
participant stored in an individual directory. A typical subject folder has the 
following structure:

\begin{verbatim}
sub-00001/
    anat/
        sub-00001_XXX-XXXXX_T1w.nii.gz
        sub-00001_XXX-XXXXX_T1w.json
        sub-00001_XXX-XXXXX_FLAIR.nii.gz
        sub-00001_XXX-XXXXX_FLAIR.json
        sub-00001_XXX-XXXXX_FLAIR_roi.nii.gz     (patients only)
\end{verbatim}

Each NIfTI image is accompanied by a corresponding JSON sidecar file containing 
DICOM-derived metadata, including voxel size, TR/TE/TI parameters, image 
orientation, and scanner information.

At the dataset root, several metadata files are provided:

\begin{verbatim}
participants.tsv
participants.json
participants_with_scanner.tsv
dataset_description.json
\end{verbatim}

The file \texttt{participants.tsv} contains demographic and clinical information 
(e.g., age, sex, diagnosis, lesion laterality, histopathological subtype), 
while \texttt{participants.json} provides descriptions of these variables. 
The file \texttt{participants\_with\_scanner.tsv} includes additional scanner-related 
metadata such as field strength and vendor. 
The standard BIDS file \\ \texttt{dataset\_description.json} documents the dataset-level 
metadata, and \texttt{meld\_bids\_config.json} contains configuration parameters 
required for MELD preprocessing and quality control.

Manually delineated lesion masks (\texttt{*\_roi.nii.gz}) are available only for 
patients and are spatially aligned with the anatomical T1-weighted MRI.

\section{MELD Dataset}
The \textbf{\ac{meld} Project} provides a large-scale neuroimaging dataset comprising \ac{mri} scans 
and clinical data of patients with FCD, as well as healthy controls. 
In total, the dataset includes \textbf{1185 participants} from \textbf{23 international epilepsy 
surgery centers}, including the Bonn datasets described earlier. 
Due to missing or corrupted files in the obtained release, we used a subset of 
\textbf{1130 participants} in our experiments. 
The dataset is not publicly available; access must be requested directly from the study authors~\cite{Ripart2025MELD}.


For each participant, the dataset includes:
\begin{itemize}
    \item Structural \ac{mri}: \textbf{3D T1-weighted} (acquired for all participants) and \textbf{3D FLAIR}, which was included only 
    for participants for whom this sequence was available;
    \item Lesion annotations: \textbf{manually delineated \ac{roi}} identifying FCD lesions (patients only).
    For MRI-negative cases, i.e., patients in whom no visible lesion is detected on conventional MRI despite histologically confirmed FCD, \textbf{postsurgical resection cavities} were used to guide ROI definition;

    \item Demographic and clinical metadata;
    \item Predefined splits for training, validation, and test cohorts.
\end{itemize}

\subsection*{Dataset Structure}
To reduce inter-site variability arising from differences in scanners and acquisition protocols across centers, 
the MELD dataset provides surface-based features that have been harmonized using the ComBat method~\cite{fortin2018combat}.
The provided ComBat-harmonized \texttt{.hdf5} files do not contain the multiscale
representations used in the MELD pipeline. Instead, they store only per-vertex
cortical attributes extracted by FreeSurfer (e.g., cortical thickness,
curvature, sulcal depth, and T1/FLAIR intensity features), together with their
asymmetry maps and the corresponding lesion masks.

Clinical and demographic metadata are distributed separately in CSV/TSV files. 
The available metadata fields include: \textit{participant ID, scanning site, diagnostic group, age of onset,
epilepsy duration, age at presurgical MRI, sex, MRI-negative status, Engel
outcome, histopathology, lesion hemisphere and lobe, presence of FLAIR
hyperintensity, seizure outcome,} and scanner information.

The dataset release also includes predefined train/validation/test splits, which
we follow to ensure comparability with prior MELD studies.


\section{FreeSurfer Processing}
Each image was processed with the FreeSurfer framework~\cite{Fischl2012FreeSurfer}, from which \textbf{11 core surface-based features} were extracted:
\begin{itemize}
    \item \textbf{Morphometric features:} cortical thickness, sulcal depth, curvature, and intrinsic curvature;
    \item \textbf{Intensity features:} gray–white matter intensity contrast, and FLAIR intensity sampled at 6 intracortical and subcortical depths.
\end{itemize}

In addition to the raw measurements, features were further processed into 
\textbf{raw values}, \textbf{control-normalized features}, and \textbf{asymmetry features} 
(left vs.~right hemisphere). Cortical thickness was additionally adjusted by regressing out curvature. 
Altogether, this yielded \textbf{34 input features per participant}, computed at \textbf{163{,}842 vertices} 
on a bilaterally symmetrical cortical surface template~\cite{Ripart2025MELD}. Scanner-related variability was addressed through the ComBat harmonization applied in the released MELD features.

For conversion from volume space to FreeSurfer surface space, detailed instructions are available in the MELD documentation~\cite{MELDdocs}.
In practice, however, it is often more practical to request preprocessed files directly from the authors, 
since converting a single image to surface space is computationally expensive: approximately 6–7 hours per scan 
(even with FastSurfer~\cite{Henschel2020FastSurfer} on an NVIDIA A100 GPU, the process required 3–4 hours).

\section{Data Augmentation}
In the original study MELD authors applied augmentations in three stages: 
\begin{itemize}
    \item \textbf{Lesion--mask augmentation}: deforming the lesion region together with its corresponding lesion mask and surface-based features to produce anatomically coherent augmented samples;
    \item \textbf{Mesh--space transforms}: applying geometric transformations to the cortical surface mesh, consistently affecting both features and lesion masks;
    \item \textbf{Intensity transforms}: modifying the intensity features at each vertex.
\end{itemize}
Each transform is applied independently with probability $p$ defined in the experiment configuration. 
The order is fixed: lesion--mask augmentation $\rightarrow$ mesh transforms $\rightarrow$ intensity transforms.

\subsection*{Lesion--mask augmentation}
Given a geodesic distance map $D$ on the cortical surface (negative inside the lesion),
we first normalise it by $|\min D|$ and add low–frequency noise, which is generated on a low–resolution icosphere (level~2) and then upsampled to the target resolution using the predefined unpool operators:
\[
\tilde{D} \;=\; \texttt{Unpool}\!\left(\tfrac{D}{|\min D|} \;+\; \mathcal{N}(0,\sigma^2)\right), 
\qquad
L' \;=\; \mathbbm{1}\{\tilde{D}\le 0\}.
\]
In our implementation, we use $\sigma = 0.5$ by default. And as \texttt{Unpool} we use the \texttt{HexUnpool} operator (see Appendix~\nameref{sec:hexunpool}).
After modifying the binary lesion mask $L$, the geodesic distances and smoothed labels are 
\textbf{recomputed}:
\[
D'=\texttt{fast\_geodesics}(L'), 
\qquad
\widetilde{L}=\texttt{smoothing}(L',\,\text{iteration}=10).
\]
This procedure is applied only if the lesion mask is non-empty.  
After augmentation, the geodesic distances and smoothed labels are recomputed to ensure a consistent lesion shape. 
However, in our training setup, only the resulting binary lesion mask is used for supervision, 
while the recomputed distances and smoothed labels are not used directly in the loss function.

\subsection*{Mesh--space transforms (icosphere re–indexing)}
We use precomputed vertex index mappings provided by the MELD authors, 
which define fixed permutations of vertices on the icosphere mesh. 
These mappings are generated offline for each transformation type 
(spinning, warping, and flipping) and are applied uniformly to all vertex-wise tensors 
(features, labels, distances, etc.) to preserve anatomical correspondence (\href{https://github.com/MELDProject/meld_graph/tree/main/data}{GitHub repository}).
The MELD framework defines three types of such maps:
\begin{itemize}
  \item \textbf{Spinning} — index remapping that corresponds to a rigid rotation of the icosphere.
  \item \textbf{Warping} — smooth non-rigid remapping that locally compresses or stretches the mesh.
  \item \textbf{Flipping} — mirror reflection of the mesh, e.g.\ exchanging left and right hemispheres. 
%   \emph{Note:} because SpiralConv is order–sensitive, flipping requires reversing the spiral neighbour order; in our final experiments we set $p(\text{flip}){=}0$ to avoid this mismatch.
\end{itemize}

\subsection*{Per-vertex feature intensity transforms}
All per-vertex intensity transformations are applied to all feature channels. 
For each transformation, parameters are sampled per channel; 
unless stated otherwise, the same parameters are applied consistently across all vertices within a channel:
\begin{itemize}
\item \textbf{Gaussian noise}: add $\epsilon\sim\mathcal{N}(0,\sigma^2)$ with $\sigma^2 \sim \mathcal{U}(0,0.1)$.
\item \textbf{Brightness scaling}: multiply each channel by $m_c\sim\mathcal{U}(0.75,1.25)$.
\item \textbf{Contrast adjustment}: for each channel $c$, sample $f\sim\mathcal{U}(0.65,1.5)$ and apply
\[
x_c \;\leftarrow\; (x_c - \mu_c)\,f + \mu_c,
\]
\[min_c \coloneqq \min(x_c), \quad
max_c \coloneqq \max(x_c),
\]
\[
x_c[x_c < min_c] \;=\; min_c, \quad
x_c[x_c > max_c] \;=\; max_c.
\]
\item \textbf{Gamma}: for $\gamma\sim\mathcal{U}(0.7,1.5)$,
\[
x' \leftarrow \Big(\frac{x-\min(x)}{\max(x)-\min(x)+\varepsilon}\Big)^{\gamma}\,(\max(x)-\min(x)+\varepsilon)+\min(x);
\]
\[
x'' \;\leftarrow\; \frac{x' - \mu}{\sigma + \varepsilon},
\quad \text{where } \varepsilon = 10^{-7},\;
\mu = \mathrm{mean}(x),\;
\sigma = \mathrm{std}(x).
\]

We also use an \emph{inverted} variant by applying the same operation to $-x$ and negating back.
\end{itemize}

\subsection*{Disabled/placeholder transforms}
The current implementation contains placeholders for \texttt{Gaussian blur} and \texttt{low–resolution downsampling}.
These operations are not implemented in the original authors’ code and were not used in our reported results.

\section{Types of Atlas Descriptions}

We generated textual descriptions with \texttt{AtlasReader}~\cite{notter2019atlasreader}.
Given each ROI, it is first registered to template space and anatomical atlases are overlaid to quantify its anatomical context.
For every atlas region, we compute the percentage of the ROI volume that overlaps that region and report the resulting region–percentage table.
When visual inputs are augmented, the corresponding textual descriptions are regenerated from the transformed lesion location to remain anatomically consistent, ensuring alignment between visual and textual modalities.
In this work, we used the Harvard--Oxford (cortical and subcortical) probabilistic atlas~\cite{harvardoxford_rrid}.

However, strong validation scores obtained using full anatomical descriptions do not guarantee 
robustness at deployment: in real-world scenarios, neither precise overlap percentages nor complete 
region names are typically available. Therefore, in Chapter~\ref{chapter:Experiments} we also evaluate 
several reduced-text settings:

\begin{enumerate}[label=\Roman*.]
    \item \emph{hemisphere only} (Left / Right);
    \item \emph{fine-grained lobe regions} (e.g., Precentral Gyrus, Frontal Pole, Subcallosal Cortex);
    \item \emph{coarse lobe labels} (e.g., Frontal, Temporal, Parietal, Occipital, Insular);
    \item \emph{hemisphere + lobe / lobe regions};
    \item \emph{mixed} (randomly sampling one of 
    \{hemisphere only, lobe region only, hemisphere + lobe, full text, no text\}; 
    this scheme is described in more detail in the next chapter).
\end{enumerate}

The anatomical names of lobes and their respective regions are taken from
\href{https://www.sciencedirect.com/science/article/pii/S1053811901909784}{\textit{Automated Anatomical Labeling of Activations in SPM Using a Macroscopic Anatomical Parcellation of the MNI MRI Single-Subject Brain}}.

\end{document}
