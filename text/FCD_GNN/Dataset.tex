\documentclass[FCD_GNN.tex]{subfiles}

\begin{document}
\chapter{Dataset}
\label{chapter:Dataset}
We gratefully acknowledge the \textbf{\ac{meld} Project} for providing access to the dataset used in this work. 
Without this resource, it would not have been possible to conduct a systematic evaluation of our model.

\section{Bonn Dataset}

The Bonn scientific group has published a publicly available presurgical MRI dataset on \textbf{OpenNeuro}, entitled 
“\textit{An open presurgery MRI dataset of people with epilepsy and focal cortical dysplasia type II}”~\cite{Rueber2023_BonnFCDDataset}.  
This dataset comprises \textbf{170 participants}, including \textbf{85 individuals with FCD type II} and \textbf{85 healthy controls}. 
For each participant, the following data are available:

\begin{itemize}
    \item High-resolution \textbf{3D T1-weighted} MRI scans (isotropic voxels, 1\,mm$^{3}$ or 0.8\,mm$^{3}$ resolution, depending on the subject);
    \item Corresponding \textbf{isotropic 3D FLAIR} imaging, available for most participants;
    \item \textbf{Manually delineated regions of interest (ROIs)} identifying FCD lesions, provided for patients only;
    \item A set of \textbf{clinical and demographic variables} (including age, sex, lesion laterality and location, histopathological subtype IIa/IIb, MRI-negative status, and postoperative outcome according to Engel classification).
\end{itemize}

\subsection*{Dataset Organization}
The dataset follows the BIDS standard. Each subject folder contains an \texttt{anat} subfolder with NIfTI files and corresponding JSON metadata. 
For participants with FCD, lesion ROI masks are also provided. Sequence acquisition parameters are documented in the JSON sidecar files.  

\section{MELD Dataset}
The \textbf{\ac{meld} Project} provides a large-scale neuroimaging dataset comprising \ac{mri} scans and clinical data of patients with FCD, as well as healthy controls. 
In total, the dataset includes \textbf{1185 participants} from \textbf{23 international epilepsy surgery centers}. 
Due to missing or corrupted files in the obtained release, we used a subset of \textbf{960 participants} in our experiments. 
The dataset is not publicly available; access must be requested directly from the study authors~\cite{Ripart2025MELD}.

For each participant, the dataset includes:
\begin{itemize}
    \item Structural \ac{mri}: \textbf{3D T1-weighted} (all participants) and, when available, \textbf{3D FLAIR};
    \item Lesion annotations: \textbf{manually delineated \ac{roi}} identifying FCD lesions (patients only). For MRI-negative cases, \textbf{postsurgical resection cavities} were used to guide ROI definition;
    \item Demographic and clinical metadata;
    \item Predefined splits for training, validation, and test cohorts.
\end{itemize}

\section{FreeSurfer Processing}
Each image was processed with the FreeSurfer framework~\cite{Fischl2012FreeSurfer}, from which \textbf{11 core surface-based features} were extracted:
\begin{itemize}
    \item \textbf{Morphometric features:} cortical thickness, sulcal depth, curvature, and intrinsic curvature;
    \item \textbf{Intensity features:} gray–white matter intensity contrast, and FLAIR intensity sampled at 6 intracortical and subcortical depths.
\end{itemize}

In addition to the raw measurements, features were further processed into 
\textbf{raw values}, \textbf{control-normalized features}, and \textbf{asymmetry features} 
(left vs.~right hemisphere). Cortical thickness was additionally adjusted by regressing out curvature. 
Altogether, this yielded \textbf{34 input features per participant}, computed at \textbf{163{,}842 vertices} 
on a bilaterally symmetrical cortical surface template~\cite{Ripart2025MELD}.

Since \ac{mri} features vary across scanners and sites, potentially impairing generalization, 
an \textbf{intersite harmonization} procedure was applied using the distributed \textbf{ComBat} algorithm, 
a well-established statistical method for removing scanner/site effects~\cite{Ripart2025MELD}.

For conversion from volume space to FreeSurfer surface space, detailed instructions are available in the MELD documentation~\cite{MELDdocs}.
In practice, however, it is often more practical to request preprocessed files directly from the authors, 
since converting a single image to surface space is computationally expensive: approximately 6–7 hours per scan 
(even with FastSurfer~\cite{Henschel2020FastSurfer} on an NVIDIA A100 GPU, the process required 3–4 hours).

\section{Data Augmentation}
In the original study MELD authors applied augmentations in three stages: 
\begin{itemize}
    \item \textbf{Lesion augmentation} — modifying the lesion mask to create new training samples;
    \item \textbf{Mesh--space transforms} — applying geometric transformations to the cortical surface mesh;
    \item \textbf{Intensity transforms} — modifying the intensity features at each vertex.
\end{itemize}
Each transform is applied independently with probability $p$ defined in the experiment configuration. 
The order is fixed: lesion augmentation $\rightarrow$ mesh transforms $\rightarrow$ intensity transforms.

\paragraph{Lesion--mask augmentation.}
Given a geodesic distance map $D$ on the cortical surface (negative inside the lesion),
we first normalise it by $|\min D|$ and add low–frequency noise, which is generated on a low–resolution icosphere (level~2) and then upsampled to the target resolution using the predefined unpool operators:
\[
\tilde{D} \;=\; \texttt{Unpool}\!\left(\tfrac{D}{|\min D|} \;+\; \mathcal{N}(0,\sigma^2)\right), 
\qquad
L' \;=\; \mathbbm{1}\{\tilde{D}\le 0\}.
\]
In our implementation, we use $\sigma = 0.5$ by default. And as \texttt{Unpool} we use the \texttt{HexUnpool} operator (see Appendix~\nameref{sec:hexunpool}).
After modifying the binary lesion mask $L$, the geodesic distances and smoothed labels are 
\textbf{recomputed}:
\[
D'=\texttt{fast\_geodesics}(L'), 
\qquad
\widetilde{L}=\texttt{smoothing}(L',\,\text{iteration}=10).
\]
This procedure is applied only if the lesion mask is non–empty.  
In our configurations, however, the recomputed distances and smoothed labels were not used further.

\paragraph{Mesh--space transforms (icosphere re–indexing).}
We use precomputed index maps (provided by the MELD authors in their \href{https://github.com/MELDProject/meld_graph/tree/main/data}{GitHub repository}) and apply them once per sample to all vertex–wise tensors (features, labels, distances, etc.). 
The MELD framework defines three types of such maps:
\begin{itemize}
  \item \textbf{Spinning} — index remapping that corresponds to a rigid rotation of the icosphere.
  \item \textbf{Warping} — smooth non-rigid remapping that locally compresses or stretches the mesh.
  \item \textbf{Flipping} — mirror reflection of the mesh, e.g.\ exchanging left and right hemispheres. \emph{Note:} because SpiralConv is order–sensitive, flipping requires reversing the spiral neighbour order; in our final experiments we set $p(\text{flip}){=}0$ to avoid this mismatch.
\end{itemize}

\paragraph{Per–vertex intensity transforms.}
All intensity transforms act channel–wise on \texttt{features}:
\begin{itemize}
\item \textbf{Gaussian noise}: add $\epsilon\sim\mathcal{N}(0,\sigma^2)$ with $\sigma^2 \sim \mathcal{U}(0,0.1)$.
\item \textbf{Brightness scaling}: multiply each channel by $m_c\sim\mathcal{U}(0.75,1.25)$.
\item \textbf{Contrast adjustment}: for each channel $c$, sample $f\sim\mathcal{U}(0.65,1.5)$ and apply
\[
x_c \;\leftarrow\; (x_c - \mu_c)\,f + \mu_c,
\]
\[min_c \coloneqq \min(x_c), \quad
max_c \coloneqq \max(x_c),
\]
\[
x_c[x_c < min_c] \;=\; min_c, \quad
x_c[x_c > max_c] \;=\; max_c.
\]
\item \textbf{Gamma}: for $\gamma\sim\mathcal{U}(0.7,1.5)$,
\[
x' \leftarrow \Big(\frac{x-\min(x)}{\max(x)-\min(x)+\varepsilon}\Big)^{\gamma}\,(\max(x)-\min(x)+\varepsilon)+\min(x);
\]
\[
x'' \;\leftarrow\; \frac{x' - \mu}{\sigma + \varepsilon},
\quad \text{where } \varepsilon = 10^{-7},\;
\mu = \mathrm{mean}(x),\;
\sigma = \mathrm{std}(x).
\]

We also use an \emph{inverted} variant by applying the same operation to $-x$ and negating back.
\end{itemize}

\paragraph{Disabled/placeholder transforms.}
The current implementation contains placeholders for \texttt{Gaussian blur} and \texttt{low–resolution downsampling}.
These operations are not implemented in the original authors’ code and were not used in our reported results.

\section{Types of Atlas Descriptions}

We generated textual descriptions using \texttt{AtlasReader} ~\cite{notter2019atlasreader}, which registers the 
input (statistical) MRI map to a template and, for user-specified atlases, produces region names and coordinate tables. 
In this work we used the Harvard--Oxford (cortical and subcortical) probabilistic atlas ~\cite{harvardoxford_rrid} and the AAL atlas ~\cite{tzourio2002aal}. 

However, even strong validation results with full descriptions do not guarantee robustness in real-world settings: 
in practice, exact overlap percentages and full region names are rarely available. Therefore, in 
chapter \ref{chapter:Experiments} we additionally evaluate reduced-text settings: (i) hemisphere only (Left/Right), 
(ii) gross lobar labels only (e.g., \emph{Frontal Lobe}, \emph{Temporal Lobe}, \emph{Parietal lobe} \dots), and (iii) their combination.

\end{document}
